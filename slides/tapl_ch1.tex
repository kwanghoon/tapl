\documentclass[table]{beamer}

\usepackage{kotex}
\usepackage{amsmath}
\usepackage{amssymb}
\usepackage{verbatim}
\ifxetex
 \setsansfont{TeX Gyre Heros}
 \setsanshangulfont{맑은 고딕}
\fi
\usepackage{color, colortbl}					%tabular에서 rowcolor를 변경하기 위해서..
\usepackage{fancybox}
\usepackage{graphbox,graphicx}
\usepackage{tikz}								%오버레이에 그림을 그리기 위해서..
\usepackage{hyperref}							%하이퍼링크 처리

\usepackage{array}
\usepackage{ebproof}
\usepackage{multirow}


\usepackage{xcolor,multirow}
\usepackage{hhline}

\usepackage[T1]{fontenc}
\usepackage{textcomp}
\usepackage{listings}							%자바 코드를 위해서..
\lstset{
	mathescape,
	language=Java,
	basicstyle=\footnotesize\ttfamily,
	keywordstyle={}, % \footnotesize\color{blue}\ttfamily,
	captionpos=b,
	escapeinside=@@,
	showstringspaces=false,					%공백 문제 제거를 위해서
	tabsize=2,
	upquote=true
}

%----------------- 총 페이지수에서 백업 슬라이드를 제거하기 위해서....
\newcommand{\backupbegin}{
   \newcounter{framenumberappendix}
   \setcounter{framenumberappendix}{\value{framenumber}}
}
\newcommand{\backupend}{
   \addtocounter{framenumberappendix}{-\value{framenumber}}
   \addtocounter{framenumber}{\value{framenumberappendix}} 
}
%-----------------

%---------------- tabular에서 rowcolor를 변경하기 위해서..
\definecolor{Gray}{gray}{0.9}
%-----------------

%\usetheme{CambridgeUS}
% \usecolortheme{default}

\usetheme{default}
\usecolortheme{default}

%\usetheme{default}
%\usecolortheme{beaver}
%\usetheme{CambridgeUS}
%\usecolortheme{seagull}

%\useinnertheme{rectangles}


\setbeamertemplate{caption}{\insertcaption}	%그림의 캡션 자동 만들지 않도록.

\addtobeamertemplate{navigation symbols}{}{%
    \usebeamerfont{footline}%
    \usebeamercolor[fg]{footline}%
    \hspace{1em}%
    \insertframenumber/\inserttotalframenumber
}

\title[Types and Programming Languages]{1. Introduction \\
(Types and Programming Languages)}
\author[K. Choi]{Kwanghoon Choi}
\institute[Chonnam National University]{
Software Languages and Systems Laboratory \\
	Chonnam National University}
\date{Week 1}

%%%%% Macros %%%%%

\newcommand{\rpc}{$\lambda_{rpc}$}
\newcommand{\polyrpc}{$\lambda_{rpc}^{\forall}$}
\newcommand{\stateencrpc}{$\lambda_{rpc}^{enc}$}
\newcommand{\statefulrpc}{$\lambda_{rpc}^{state}$}

\newcommand{\cs}{$\lambda_{cs}$}
\newcommand{\stateenccs}{$\lambda_{cs}^{enc}$}
\newcommand{\statefulcs}{$\lambda_{cs}^{state}$}

\newcommand{\client}{\textbf{c}}
\newcommand{\server}{\textbf{s}}
\newcommand{\clientserver}{\textbf{cs}}

\newcommand{\Loc}{Loc}

\newcommand{\evalRPC}[3]{#1\Downarrow_{#2}#3}
\newcommand{\evalRPCC}[2]{#1\Downarrow_{\client}#2}
\newcommand{\evalRPCS}[2]{#1\Downarrow_{\server}#2}
\newcommand{\lamL}[3]{\lambda^{#1}#2.#3}
\newcommand{\appL}[3]{#1{\ }^{#2}#3} 
\newcommand{\subst}[2]{\{#1/#2\}}
\newcommand{\llet}[3]{\textsf{let} \ #1 = #2 \ \textsf{in} \ #3}

\newcommand{\textsfReq}{\textsf{req}}
\newcommand{\req}[2]{\textsfReq(#1,#2)}
\newcommand{\reqwith}[3]{\textsfReq_{#1}(#2,#3)}

\newcommand{\textsfCall}{\textsf{call}}
\newcommand{\call}[2]{\textsfCall(#1,#2)}
\newcommand{\callwith}[3]{\textsfCall_{#1}(#2,#3)}

\newcommand{\textsfRet}{\textsf{ret}}
\newcommand{\ret}[1]{\textsfRet(#1)}
\newcommand{\retwith}[2]{\textsfRet_{#1}(#2)}

\newcommand{\funL}[1]{\xrightarrow{#1}}    
\newcommand{\funLC}[2]{\xrightarrow[#2]{#1}}  
\newcommand{\tyenv}{\Gamma}     
\newcommand{\tyenvExt}[2]{\Gamma\{#1:#2\}}
\newcommand{\tyenvExtWith}[1]{\Gamma,#1}
% \newcommand{\typing}[4]{#1\rhd_{#2} #3 : #4}       
\newcommand{\typing}[4]{#1\vdash_{#2} #3 : #4}       
\newcommand{\typingBlack}[4]{#1\blacktriangleright_{#2} #3 : #4}  

\newcommand{\loceta}[2]{{#1}\rightsquigarrow{#2}}                                         

\newcommand{\enc}{\textsf{enc}}
\newcommand{\evalStateEncRPCC}[2]{$#1\Downarrow_{\client}^{\enc}#2$}
\newcommand{\evalStateEncRPCS}[3]{$#1;#2\Downarrow_{\server}^{\enc}#3$}

\newcommand{\sta}{\textsf{state}}
\newcommand{\evalStatefulRPCC}[3]{\evalStatefulRPC{#1}{#2}{\client}{#3}}
\newcommand{\evalStatefulRPCS}[3]{\evalStatefulRPC{#1}{#2}{\server}{#3}}
\newcommand{\evalStatefulRPC}[4]{${#1};{#2}\Downarrow_{#3}^{\sta}{#4}$}

\newcommand{\deep}{\textsf{deep}}
\newcommand{\evalDeeplyStatefulRPCC}[3]{\evalDeeplyStatefulRPC{#1}{#2}{\client}{#3}}
\newcommand{\evalDeeplyStatefulRPCS}[3]{\evalDeeplyStatefulRPC{#1}{#2}{\server}{#3}}
\newcommand{\evalDeeplyStatefulRPC}[4]{${#1};{#2}\Downarrow_{#3}^{\deep}{#4}$}

\newcommand{\IdK}{\textsf{Id}}
\newcommand{\FunK}[3]{\textsf{Fun} \ #2 \ #3}
\newcommand{\AppK}[3]{\textsf{App} \ #2 \ #3}
%\newcommand{\FunK}[3]{\textsf{Fun}^{#1} \ #2 \ #3}
%\newcommand{\AppK}[3]{\textsf{App}^{#1} \ #2 \ #3}
                                                
\newcommand{\reify}[1]{\ulcorner #1 \urcorner}

\newcommand{\RightarrowEnc}{\Rightarrow^{enc}}
\newcommand{\RightarrowEncStar}{\Rightarrow^{enc*}}
\newcommand{\RightarrowEncPlus}{\Rightarrow^{enc+}}

\newcommand{\runStateEncRPC}[2]{$#1 \RightarrowEnc #2$}
\newcommand{\runStateEncRPCStar}[2]{$#1 \RightarrowEncStar #2$}
\newcommand{\runStateEncRPCPlus}[2]{$#1 \RightarrowEncPlus #2$}

\newcommand{\runStateEncCS}[2]{$#1 \Rightarrow^{enc} #2$}
\newcommand{\runStateEncCSStar}[2]{$#1 \Rightarrow^{enc*} #2$}

\newcommand{\runStatefulRPC}[2]{$#1 \Rightarrow^{state} #2$}
\newcommand{\runStatefulRPCStar}[2]{$#1 \Rightarrow^{state*} #2$}

\newcommand{\runStatefulCS}[2]{$#1 \Rightarrow^{state} #2$}
\newcommand{\runStatefulCSStar}[2]{$#1 \Rightarrow^{state*} #2$}

\newcommand{\emp}{\epsilon}
\newcommand{\substzsxs}{\{\bar{v}/\bar{z},\overline{w}/\bar{x} \}}
\newcommand{\substxs}{\{\overline{w}/\bar{x} \}}

\newcommand{\substZsXs}{\{\overline{V}/\bar{z},\overline{W}/\bar{x} \}}
\newcommand{\substXs}{\{\overline{W}/\bar{x} \}}

\newcommand{\LetK}[2]{\textsf{ctx}\ #1 \ #2}
%\newcommand{\LetK}[2]{(#1,#2)}
\newcommand{\opt}[1]{#1_{opt}}

\newcommand{\overlineK}{\overline{K}}
\newcommand{\overlinePi}{\overline{\Pi}}
\newcommand{\overlineDelta}{\overline{\Delta}}

%\newcommand{\comp}[1]{\rightsquigarrow_{#1}}
%\newcommand{\comps}{\rightsquigarrow_{\server}}
%\newcommand{\compc}{\rightsquigarrow_{\client}}
\newcommand{\ccomp}[1]{C[\![#1]\!]}
\newcommand{\scomp}[1]{S[\![#1]\!]}
\newcommand{\vcomp}[1]{V[\![#1]\!]}
\newcommand{\cconv}[1]{CC[\![#1]\!]}
%\newcommand{\cconvprg}[1]{CC_{prg}[\![#1]\!]}

\newcommand{\FUNS}{\Phi}
\newcommand{\fv}[1]{\textsf{fv}(#1)}
\newcommand{\dom}[1]{\textsf{dom}(#1)}
\newcommand{\clo}[2]{clo({#1},{#2})}

\newcommand{\sessionNothing}{\makebox[0.3cm][c]{\scriptsize $nothing$}}
\newcommand{\sessionSomething}{\makebox[0.3cm][c]{\scriptsize $session$}}
\newcommand{\sessionOption}{\makebox[0.3cm][c]{\scriptsize $optSession$}}

\newcommand{\mono}[1]{[\![#1]\!]}

\newcommand{\eqdef}{\overset{\mathrm{def}}{=\joinrel=}}

%%
%\newtheorem{lemma}{Lemma}[section]
%\newtheorem{theorem}{Theorem}[section]
%\newtheorem{fact}{Fact}[section]
%\newtheorem{definition}{Definition}[section]

%%%%%%%%%%%%%%%%%%

\begin{document}

%\section{Relative Pronoun}
\begin{frame}
%\begin{center}
%{\footnotesize
%Types and Programming Languages
%}
%\end{center}
	\titlepage
	
%	\begin{center}
%	{
%  (Joint work with James Cheney, Simon Fowler, and Sam Lindley)
%	}
%	\end{center}
\end{frame}

%\begin{frame}{Table of Contents}
%\begin{itemize}
%\item
%\item
%\item
%\item
%\end{itemize}
%\end{frame}

\begin{frame}[t]{1.1 Background: Software Engineering} \vspace{10pt}

To make software correctly work and to make it smoothly evolve are very difficult and very expensive.

\vspace{10pt}

Two main approaches to tame  software monsters are 
\begin{itemize}
\item empirical software engineering based on {\it experiences}

\item {\it formal methods}  based on {\it mathematics and logics}

\end{itemize}

\vspace{10pt}

In this class, we are interested in formal methods so called {\it type systems}. 

\end{frame}

%%%
\begin{frame}[t]{1.1 Background: Formal Methods} \vspace{10pt}


A broad range of {\it formal methods } in software engineering

\vspace{10pt}

Some powerful frameworks at one end of the spectrum

\begin{itemize}

\item  Hoare logic, Algebraic specification languages, Modal logics, and Denotational semantics

\end{itemize}

\vspace{10pt}

Pros: They can express very general correctness properties 

\vspace{10pt}

Cons: 
\begin{itemize}
\item But they are often cumbersome to use and demand a good deal of sophistication on the part of programmers
\end{itemize}

\end{frame}

%%%
\begin{frame}[t]{1.1 Background: Formal Methods (cont.)} \vspace{10pt}

A broad range of {\it formal methods } in software engineering 

\vspace{10pt}

{\it Lightweight formal methods} at the other end of the spectrum
\begin{itemize}
\item Model checkers, Run-time monitoring, {\it Type systems}, etc.
\end{itemize}

\vspace{10pt}

Cons: 
\begin{itemize}
\item Much more modest power than the previous powerful frameworks
\end{itemize}

\vspace{10pt}

Pros: 
\begin{itemize}
\item But modest enough that automatic checkers can be built into compilers  
\item and thus be applied even by programmers unfamiliar with the underlying theories
\end{itemize}

\end{frame}

%%%
\begin{frame}[t]{1.1 Types in Computer Science} \vspace{10pt}

In this class, a type system is 
\begin{itemize}
\item a tractable syntactic method
\item for proving the absence of certain program behaviors 
\item by classifying phrases according to the kinds of values they compute
\end{itemize}

 \vspace{10pt}
 
 $\Rightarrow$ Type systems as tools for reasoning about programs.
 \vspace{5pt}
 
  $\Rightarrow$ Type systems as calculating a kind of {\it static approximation} to the certain {run-time behaviors of the terms}  in a program.
  
 \vspace{10pt}
 
 But type systems (or {\it type theory}) refers to a much broader field of study in logic, mathematics, and philosophy, dating back to the early 1900s as ways of avoiding the logical paradoxes, such as Russell's. 
 
\end{frame}

%%%
\begin{frame}[t]{1.2 What Type Systems Are Good For} \vspace{10pt}

Detecting errors

\vspace{10pt}

Abstraction

\vspace{10pt}

Documentation

\vspace{10pt}

Language safety

\vspace{10pt}

Efficiency

\vspace{10pt}

Further applications : Automated theorem proving such as with Coq, Isabelle, Agda
\end{frame}


%%%
\begin{frame}[t]{1.3 Type Systems and Language Design} \vspace{10pt}

Programming language design go hand in hand with type-system design

\vspace{10pt}

The type system itself is the foundation of the design and the organizing principle in typed programming languages

\vspace{10pt}

Even in untyped programming languages, they are structured in a way for which a type system can be written.

\vspace{10pt}

In summary, there are no programming languages without type systems!

%\vspace{10pt}
%
%Trade-off between explicit type annotation vs. automatic type inference

\end{frame}

%%%
\begin{frame}[t]{1.4 Capsule History} \vspace{10pt}
The earliest type systems to make very simple distinctions between integer and floating point representations of numbers in Fortran (1950s)

\vspace{10pt}

This classification was extended to structured data (arrays, records, etc.) and to higher-order functions (1960s)

\vspace{10pt}

A number of even richer concepts (parametric polymorphism, abstract data types, module systems, and subtyping) were introduced (1970s).

\vspace{10pt}

Computer scientists began to be aware of the connections between the type systems found in PLs and those studied in mathematical logic! 

\end{frame}

%%%
\begin{frame}[t]{1.5 Related Reading} \vspace{10pt}

Many useful, interesting, and mind-blowing textbooks about types and programming languages 

\end{frame}

\end{document}



