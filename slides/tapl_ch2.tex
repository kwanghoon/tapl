\documentclass[table]{beamer}

\usepackage{kotex}
\usepackage{amsmath}
\usepackage{amssymb}
\usepackage{verbatim}
\ifxetex
 \setsansfont{TeX Gyre Heros}
 \setsanshangulfont{맑은 고딕}
\fi
\usepackage{color, colortbl}					%tabular에서 rowcolor를 변경하기 위해서..
\usepackage{fancybox}
\usepackage{graphbox,graphicx}
\usepackage{tikz}								%오버레이에 그림을 그리기 위해서..
\usepackage{hyperref}							%하이퍼링크 처리

\usepackage{array}
\usepackage{ebproof}
\usepackage{multirow}


\usepackage{xcolor,multirow}
\usepackage{hhline}

\usepackage[T1]{fontenc}
\usepackage{textcomp}
\usepackage{listings}							%자바 코드를 위해서..
\lstset{
	mathescape,
	language=Java,
	basicstyle=\footnotesize\ttfamily,
	keywordstyle={}, % \footnotesize\color{blue}\ttfamily,
	captionpos=b,
	escapeinside=@@,
	showstringspaces=false,					%공백 문제 제거를 위해서
	tabsize=2,
	upquote=true
}

\usepackage{fdsymbol}

%----------------- 총 페이지수에서 백업 슬라이드를 제거하기 위해서....
\newcommand{\backupbegin}{
   \newcounter{framenumberappendix}
   \setcounter{framenumberappendix}{\value{framenumber}}
}
\newcommand{\backupend}{
   \addtocounter{framenumberappendix}{-\value{framenumber}}
   \addtocounter{framenumber}{\value{framenumberappendix}} 
}
%-----------------

%---------------- tabular에서 rowcolor를 변경하기 위해서..
\definecolor{Gray}{gray}{0.9}
%-----------------

%\usetheme{CambridgeUS}
% \usecolortheme{default}

\usetheme{default}
\usecolortheme{default}

%\usetheme{default}
%\usecolortheme{beaver}
%\usetheme{CambridgeUS}
%\usecolortheme{seagull}

%\useinnertheme{rectangles}


\setbeamertemplate{caption}{\insertcaption}	%그림의 캡션 자동 만들지 않도록.

\addtobeamertemplate{navigation symbols}{}{%
    \usebeamerfont{footline}%
    \usebeamercolor[fg]{footline}%
    \hspace{1em}%
    \insertframenumber/\inserttotalframenumber
}

\title[Types and Programming Languages]{2. Mathematical Preliminaries \\
(Types and Programming Languages)}
\author[K. Choi]{Kwanghoon Choi}
\institute[Chonnam National University]{
Software Languages and Systems Laboratory \\
	Chonnam National University}
\date{Week 2}

%%%%% Macros %%%%%

\newcommand{\rpc}{$\lambda_{rpc}$}
\newcommand{\polyrpc}{$\lambda_{rpc}^{\forall}$}
\newcommand{\stateencrpc}{$\lambda_{rpc}^{enc}$}
\newcommand{\statefulrpc}{$\lambda_{rpc}^{state}$}

\newcommand{\cs}{$\lambda_{cs}$}
\newcommand{\stateenccs}{$\lambda_{cs}^{enc}$}
\newcommand{\statefulcs}{$\lambda_{cs}^{state}$}

\newcommand{\client}{\textbf{c}}
\newcommand{\server}{\textbf{s}}
\newcommand{\clientserver}{\textbf{cs}}

\newcommand{\Loc}{Loc}

\newcommand{\evalRPC}[3]{#1\Downarrow_{#2}#3}
\newcommand{\evalRPCC}[2]{#1\Downarrow_{\client}#2}
\newcommand{\evalRPCS}[2]{#1\Downarrow_{\server}#2}
\newcommand{\lamL}[3]{\lambda^{#1}#2.#3}
\newcommand{\appL}[3]{#1{\ }^{#2}#3} 
\newcommand{\subst}[2]{\{#1/#2\}}
\newcommand{\llet}[3]{\textsf{let} \ #1 = #2 \ \textsf{in} \ #3}

\newcommand{\textsfReq}{\textsf{req}}
\newcommand{\req}[2]{\textsfReq(#1,#2)}
\newcommand{\reqwith}[3]{\textsfReq_{#1}(#2,#3)}

\newcommand{\textsfCall}{\textsf{call}}
\newcommand{\call}[2]{\textsfCall(#1,#2)}
\newcommand{\callwith}[3]{\textsfCall_{#1}(#2,#3)}

\newcommand{\textsfRet}{\textsf{ret}}
\newcommand{\ret}[1]{\textsfRet(#1)}
\newcommand{\retwith}[2]{\textsfRet_{#1}(#2)}

\newcommand{\funL}[1]{\xrightarrow{#1}}    
\newcommand{\funLC}[2]{\xrightarrow[#2]{#1}}  
\newcommand{\tyenv}{\Gamma}     
\newcommand{\tyenvExt}[2]{\Gamma\{#1:#2\}}
\newcommand{\tyenvExtWith}[1]{\Gamma,#1}
% \newcommand{\typing}[4]{#1\rhd_{#2} #3 : #4}       
\newcommand{\typing}[4]{#1\vdash_{#2} #3 : #4}       
\newcommand{\typingBlack}[4]{#1\blacktriangleright_{#2} #3 : #4}  

\newcommand{\loceta}[2]{{#1}\rightsquigarrow{#2}}                                         

\newcommand{\enc}{\textsf{enc}}
\newcommand{\evalStateEncRPCC}[2]{$#1\Downarrow_{\client}^{\enc}#2$}
\newcommand{\evalStateEncRPCS}[3]{$#1;#2\Downarrow_{\server}^{\enc}#3$}

\newcommand{\sta}{\textsf{state}}
\newcommand{\evalStatefulRPCC}[3]{\evalStatefulRPC{#1}{#2}{\client}{#3}}
\newcommand{\evalStatefulRPCS}[3]{\evalStatefulRPC{#1}{#2}{\server}{#3}}
\newcommand{\evalStatefulRPC}[4]{${#1};{#2}\Downarrow_{#3}^{\sta}{#4}$}

\newcommand{\deep}{\textsf{deep}}
\newcommand{\evalDeeplyStatefulRPCC}[3]{\evalDeeplyStatefulRPC{#1}{#2}{\client}{#3}}
\newcommand{\evalDeeplyStatefulRPCS}[3]{\evalDeeplyStatefulRPC{#1}{#2}{\server}{#3}}
\newcommand{\evalDeeplyStatefulRPC}[4]{${#1};{#2}\Downarrow_{#3}^{\deep}{#4}$}

\newcommand{\IdK}{\textsf{Id}}
\newcommand{\FunK}[3]{\textsf{Fun} \ #2 \ #3}
\newcommand{\AppK}[3]{\textsf{App} \ #2 \ #3}
%\newcommand{\FunK}[3]{\textsf{Fun}^{#1} \ #2 \ #3}
%\newcommand{\AppK}[3]{\textsf{App}^{#1} \ #2 \ #3}
                                                
\newcommand{\reify}[1]{\ulcorner #1 \urcorner}

\newcommand{\RightarrowEnc}{\Rightarrow^{enc}}
\newcommand{\RightarrowEncStar}{\Rightarrow^{enc*}}
\newcommand{\RightarrowEncPlus}{\Rightarrow^{enc+}}

\newcommand{\runStateEncRPC}[2]{$#1 \RightarrowEnc #2$}
\newcommand{\runStateEncRPCStar}[2]{$#1 \RightarrowEncStar #2$}
\newcommand{\runStateEncRPCPlus}[2]{$#1 \RightarrowEncPlus #2$}

\newcommand{\runStateEncCS}[2]{$#1 \Rightarrow^{enc} #2$}
\newcommand{\runStateEncCSStar}[2]{$#1 \Rightarrow^{enc*} #2$}

\newcommand{\runStatefulRPC}[2]{$#1 \Rightarrow^{state} #2$}
\newcommand{\runStatefulRPCStar}[2]{$#1 \Rightarrow^{state*} #2$}

\newcommand{\runStatefulCS}[2]{$#1 \Rightarrow^{state} #2$}
\newcommand{\runStatefulCSStar}[2]{$#1 \Rightarrow^{state*} #2$}

\newcommand{\emp}{\epsilon}
\newcommand{\substzsxs}{\{\bar{v}/\bar{z},\overline{w}/\bar{x} \}}
\newcommand{\substxs}{\{\overline{w}/\bar{x} \}}

\newcommand{\substZsXs}{\{\overline{V}/\bar{z},\overline{W}/\bar{x} \}}
\newcommand{\substXs}{\{\overline{W}/\bar{x} \}}

\newcommand{\LetK}[2]{\textsf{ctx}\ #1 \ #2}
%\newcommand{\LetK}[2]{(#1,#2)}
\newcommand{\opt}[1]{#1_{opt}}

\newcommand{\overlineK}{\overline{K}}
\newcommand{\overlinePi}{\overline{\Pi}}
\newcommand{\overlineDelta}{\overline{\Delta}}

%\newcommand{\comp}[1]{\rightsquigarrow_{#1}}
%\newcommand{\comps}{\rightsquigarrow_{\server}}
%\newcommand{\compc}{\rightsquigarrow_{\client}}
\newcommand{\ccomp}[1]{C[\![#1]\!]}
\newcommand{\scomp}[1]{S[\![#1]\!]}
\newcommand{\vcomp}[1]{V[\![#1]\!]}
\newcommand{\cconv}[1]{CC[\![#1]\!]}
%\newcommand{\cconvprg}[1]{CC_{prg}[\![#1]\!]}

\newcommand{\FUNS}{\Phi}
\newcommand{\fv}[1]{\textsf{fv}(#1)}
\newcommand{\dom}[1]{\textsf{dom}(#1)}
\newcommand{\clo}[2]{clo({#1},{#2})}

\newcommand{\sessionNothing}{\makebox[0.3cm][c]{\scriptsize $nothing$}}
\newcommand{\sessionSomething}{\makebox[0.3cm][c]{\scriptsize $session$}}
\newcommand{\sessionOption}{\makebox[0.3cm][c]{\scriptsize $optSession$}}

\newcommand{\mono}[1]{[\![#1]\!]}

\newcommand{\eqdef}{\overset{\mathrm{def}}{=\joinrel=}}

%%
%\newtheorem{lemma}{Lemma}[section]
%\newtheorem{theorem}{Theorem}[section]
%\newtheorem{fact}{Fact}[section]
%\newtheorem{definition}{Definition}[section]

%%%%%%%%%%%%%%%%%%

\begin{document}

%\section{Relative Pronoun}
\begin{frame}
%\begin{center}
%{\footnotesize
%Types and Programming Languages
%}
%\end{center}
	\titlepage
	
%	\begin{center}
%	{
%  (Joint work with James Cheney, Simon Fowler, and Sam Lindley)
%	}
%	\end{center}
\end{frame}

%\begin{frame}{Table of Contents}
%\begin{itemize}
%\item
%\item
%\item
%\item
%\end{itemize}
%\end{frame}

%%%
\begin{frame}[t]{2.1 Sets, Relations, and Functions: Sets} \vspace{10pt}

Standard notation for sets 
\begin{itemize}
\item by enumeration, e.g., as $\{ \ 0, 2, 4, 6, \cdots  \ \}$, or 
\item by {\it comprehension}, e.g., as $\{ x \in \mathbb{N} \ | \ x \ is \ even \}$ where $\mathbb{N}$ is the set of natural numbers $\{0,1,2,3,\cdots\}$.
\end{itemize}

\vspace{10pt}

$\emptyset$ for the empty set
\vspace{10pt}

$S \backslash T$ for the set difference of $S$ and $T$
\vspace{10pt}

$|S|$ for the size of a set S
\vspace{10pt}

$\mathcal{P}(S)$ for the powerset of S, i.e., the set of all the subsets of $S$. 

\begin{itemize}
\item Q. $\mathcal{P}(\{1,2,3\}) = $
\end{itemize}

\end{frame}

%%%
\begin{frame}[t]{2.1 Sets, Relations, and Functions: Relations} \vspace{10pt}

{\color{red} An $n$-place {\it relation} over sets $S_1$, $S_2$, ... , $S_n$} is a set $R\subseteq S_1\times S_2 \times \cdots \times S_n$ of tuples of elements from $S_1$ through $S_n$. 
\begin{itemize}
\item We say that the elements $s_1\in S_1$ through $s_n\in S_n$ are {\it \color{red} related by $R$} if the tuple $(s_1, \cdots, s_n) \in R$. 
\end{itemize}

\vspace{10pt}

Q. By $MyRelation = \{ (1,2), (2,4), (3,6), (4,8), (5,10), \cdots \}$, how the elements are related?

\vspace{10pt}

Q. Define your relation and explain how elements are related by your relation?


\end{frame}

%%%
\begin{frame}[t]{2.1 Sets, Relations, and Functions: Relations} % \vspace{10pt}

A 1-place relation on a set S is called {\it \color{red} a predicate on $S$}. 
\begin{itemize}
\item $Even=\{(0),(2),(4),\cdots\}\simeq\{0,2,4,\cdots\}\subseteq \mathbb{N}$ is a relation.
\item $Even$ is regarded as a predicate on $\mathbb{N}$ as $Even(elem)$ is true iff $elem\in Even$. E.g., $Even(2)$ is true, $Even(3)$ is false, and so on.
\end{itemize}

\vspace{10pt}

A 2-place relation on $S$ and $T$ is called {\it \color{red} a binary relation}. We often write {\color{red} $s \ R \ t$} instead of $(s,t)\in R$.

\vspace{10pt}

A 2-place relation on the same sets $S$ and $S$ is simply called {\it \color{red} a binary relation on $S$}.

\vspace{10pt}

{\color{red} For readability}, three- or more place relations are often written using a ``mixfix'' concrete syntax. For example,
for the typing relation for the simply typed lambda calculus in Ch.9, we write {\color{red} $\tyenv \vdash s : T$} to mean ``the triple $(\tyenv,s,T)$ is in the typing relation.'' \\
E.g., $(x:int, x+1, int) \in TypeSystem$ but $(x:bool, x+1, ?) \not\in TypeSystem$

\end{frame}

%%%
\begin{frame}[t]{2.1 Sets, Relations, and Functions: Functions} \vspace{10pt}

Suppose a relation $R$ on sets $S$ and $T$. The {\it \color{red} domain} of $R$, written $dom(R)$, is $\{ s \ | \ (s,t)\in R\}$, and the {\it \color{red} range} of $R$, written $range(R)$, is $\{ t \ | \ (s,t)\in R\}$. E.g., $dom(MyRelation)$ and $range(MyRelation)$. 

\vspace{10pt}

Such a relation $R$ is called a {\it \color{red} (partial) function} from $S$ to $T$ if $(s,t_1)\in R$ and $(s,t_2)\in R$, we have $t_1 = t_2$. 

\vspace{10pt}

Such a relation $R$ is called a {\it \color{red} total function} if $R$ is a partial function and $dom(R)=S$. 

\vspace{10pt}

A partial function $R$ is is {\it defined} on $s$ if $s \in dom(R)$, which is denoted by {\color{red} $R(x) \downarrow$}. Otherewise it is {undefined}, denoted by {\color{red} $R(x) \uparrow$}.

\vspace{10pt}

Q. Show an example of a partial function that is not a total one.

\end{frame}


%%%
\begin{frame}[t]{2.2 Ordered Sets} % \vspace{10pt}

Suppose a binary relation $R$ on a set $S$. 

\vspace{10pt}

Q. Find out three definitions that $R$ is {\it reflexive}, $R$ is {\it symmetric}, and $R$ is {\it transitive}, respectively. 

\vspace{10pt}

$R$ is an {\it equivalence} on $S$ if it is reflexive, symmetric, and transitive. 

\vspace{10pt}

The {\it reflexive closure} of $R$ is $R \cup \{ (s,s) \ | \ s \in S \}$.

\vspace{10pt}

The {\it transitive closure} of $R$ is $R^+= \bigcup_i R_i$ where 
\begin{eqnarray*}
R_0 & = & R \\
R_{i+1} & = & R_i \cup \{ (s,u) \ | \ (s,t) \in R_i and (t,u) \in R_i \ for \ some \ t \ \}
\end{eqnarray*}

%\vspace{10pt}

The {\it reflexive and transitive closure} of $R$ is $R^*$.

\vspace{10pt}

Q. $MyRelation^*$

\end{frame}

%%%
\begin{frame}[t]{2.2 Ordered Sets} \vspace{10pt}

A predicate $P$ on a set $S$ is {\it \color{red} preserved by} a binary relation $R$ on $S$
\begin{eqnarray*}
P & : & S\\
R & : & S \times S 
\end{eqnarray*}
for all $s_1 \ R \ s_2$, if $\mathcal{P}(s_1)$ then $\mathcal{P}(s_2)$.
\vspace{10pt}

Q. Show a concrete example of $P$ and $S$. 

\vspace{10pt}

Q. Show that if $P$ is preserved by $R$, then $P$ is also preserved by $R^*$. 

\end{frame}

%%%
\begin{frame}[t]{2.3 Sequences} \vspace{10pt}

A {\it \color{red} sequence} is written by listing its elements, separated by commas, such as 1,2,3.

\vspace{10pt}

Suppose $a$ is 3,2,1 and $b$ is 5,6.
\begin{itemize}
\item 0,a denotes 0,3,2,1
\item a,0 denotes 3,2,1,0
\item b,a denotes 5,6,3,2,1
\end{itemize}

\vspace{10pt}

1..$n$ is the sequence of numbers from 1 to $n$

\vspace{10pt}

$|a|$ is the length of the sequence $a$. 

\vspace{10pt}

The empty sequence is written either as $\bullet$ or as a blank. 

\vspace{10pt}

Q. Write down all {\it \color{red} permutation}s of the sequence 1,2,3. 

\end{frame}

%%%
\begin{frame}[t]{2.4 Induction} \vspace{10pt}

To prove the infinite cases of a predicate, principles of induction may be useful.

\vspace{10pt}


[{\it \color{red} Principle of Induction on Natural Numbers}]
Suppose a predicate $\mathcal{P}$ on the natural numbers $\mathbb{N}$. 
\begin{itemize}
\item[] If $\mathcal{P}(0)$
\item[] and, for all $i\in \mathbb{N}$, $\mathcal{P}(i)$ implies $\mathcal{P}(i+1)$,
\item[] then $\mathcal{P}(n)$ is true for all $n \in \mathbb{N}$.
\end{itemize}

\end{frame}

%%%
\begin{frame}[t]{2.4 Induction} 

To prove ``infinite cases'', {\it principles of induction} may be useful.

\vspace{10pt}

Q. Let $\mathcal{P}_j\ \triangleeq \ \sum_{i=0}^j i = \frac{j(j+1)}{2}$.  
	We want to show $\mathcal{P}_j$ is true for all $j\in\mathbb{N}$.
\begin{itemize}
\item[1] (Base case) Prove $\mathcal{P}_0$ is true.
\item[2-1] Assume $\mathcal{P}_4$ is true. Then show $\mathcal{P}_5$ is provable using the assumption.
\item[2-2] (Inductive case) Generalize (2-1). Assume $\mathcal{P}_k$ is true for an arbitrary natural number $k$. Then show $\mathcal{P}_{k+1}$ is true using the assumption.
\item[3] Verify you have proved $\mathcal{P}_j$ for all (infinite cases) $j\in\mathbb{N}$ by (1) and (2-2). 
\end{itemize}
%\[
%\mathcal{P} = \left\{ k  \in \mathbb{N} \ | \ \sum_{i=0}^k i = \frac{k(k+1)}{2} \ is \ true \right\}
%\]. In other words, $\mathcal{P}(n) = \ \ \left( \sum_{i=0}^n i = \frac{n(n+1)}{2}\right) $ for all $n\in \mathbb{N}$.
%\begin{itemize}
%\item Prove $\sum_{i=0}^n i = \frac{n(n+1)}{2}$ for all $n\in\mathbb{N}$ by the principle of induction on natural numbers.
%\end{itemize}

\vspace{10pt}

It is enough to prove the base and the inductive cases rather than to prove the infinite ones!

\end{frame}

%%%
\begin{frame}[t]{2.4 Induction} \vspace{10pt}

[{\it \color{red} Principle of Induction on Natural Numbers}]
Suppose a predicate $\mathcal{P}$ on the natural numbers $\mathbb{N}$. 
\begin{itemize}
\item[] If $\mathcal{P}(0)$
\item[] and, for all $i\in \mathbb{N}$, $\mathcal{P}(i)$ implies $\mathcal{P}(i+1)$,
\item[] then $\mathcal{P}(n)$ is true for all $n \in \mathbb{N}$.
\end{itemize}

\vspace{10pt}

Q. Explain why a proof in the previous exercise can be viewed as the principle of induction.

\end{frame}

%%%
\begin{frame}[t]{2.4 Induction}

Sometimes we need another principle stronger than the ``plain'' principle of induction. 

\vspace{10pt}

Q. We want to show that any positive number integer $n$ greater than or equal to 2 is either a prime or a product of primes.

\begin{itemize}
\item Let $\mathcal{P}_j$ be either $j$ is a prime or $j$ is a product of primes.
\item[1] (Base case) Show $\mathcal{P}_2$ is true.
\item[2-1] Assume $\mathcal{P}_{16}$ and $\mathcal{P}_{64}$ are true. Then show $\mathcal{P}_{1024}$ is provable using the assumptions.
\item[2-2] (Inductive case) Generalize (2-1). Assume $\mathcal{P}_i$ is true for an arbitrary natural number $i\leq k$. Then show $\mathcal{P}_{k+1}$ is true using the assumptions.
\item[3] Verify you have proved $\mathcal{P}_j$ for all (infinite cases) $2\leq j$ by (1) and (2-2).
\end{itemize}

\vspace{10pt}

To prove $\mathcal{P}_{k+1}$, not only we may need $\mathcal{P}_{k}$ but we may also need more as: $\mathcal{P}_k$, $\mathcal{P}_{k-1}$, $\mathcal{P}_{k-2}$, $\mathcal{P}_{k-3}$, ... , $\mathcal{P}_{2}$.

\end{frame}

%%%
\begin{frame}[t]{2.4 Induction} \vspace{10pt}

[{\it \color{red} Principle of the complete induction on Natural Numbers}]
Suppose a predicate $\mathcal{P}$ on the natural numbers $\mathbb{N}$. 
\begin{itemize}
\item[] If $\mathcal{P}(0)$
\item[] and, for all $i\in \mathbb{N}$, {\color{red} $\mathcal{P}(j)$ for all $j\leq i$} implies $\mathcal{P}(i+1)$,
\item[] then $\mathcal{P}(n)$ is true for all $n \in \mathbb{N}$.
\end{itemize}

\vspace{10pt}

More induction principles exist, for example, as
\begin{itemize}
\item the principle of lexicographic induction 
\item the principle of structural induction
\end{itemize}

\end{frame}



\end{document}

