\documentclass[table]{beamer}

\usepackage{kotex}
\usepackage{amsmath}
\usepackage{amssymb}
\usepackage{verbatim}
\ifxetex
 \setsansfont{TeX Gyre Heros}
 \setsanshangulfont{맑은 고딕}
\fi
\usepackage{color, colortbl}					%tabular에서 rowcolor를 변경하기 위해서..
\usepackage{fancybox}
\usepackage{graphbox,graphicx}
\usepackage{tikz}								%오버레이에 그림을 그리기 위해서..
\usepackage{hyperref}							%하이퍼링크 처리

\usepackage{array}
\usepackage{ebproof}
\usepackage{multirow}


\usepackage{xcolor,multirow}
\usepackage{hhline}

\usepackage[T1]{fontenc}
\usepackage{textcomp}
\usepackage{listings}							%자바 코드를 위해서..
\lstset{
	mathescape,
	language=Java,
	basicstyle=\footnotesize\ttfamily,
	keywordstyle={}, % \footnotesize\color{blue}\ttfamily,
	captionpos=b,
	escapeinside=@@,
	showstringspaces=false,					%공백 문제 제거를 위해서
	tabsize=2,
	upquote=true
}

\usepackage{soul}

%----------------- 총 페이지수에서 백업 슬라이드를 제거하기 위해서....
\newcommand{\backupbegin}{
   \newcounter{framenumberappendix}
   \setcounter{framenumberappendix}{\value{framenumber}}
}
\newcommand{\backupend}{
   \addtocounter{framenumberappendix}{-\value{framenumber}}
   \addtocounter{framenumber}{\value{framenumberappendix}} 
}
%-----------------

%---------------- tabular에서 rowcolor를 변경하기 위해서..
\definecolor{Gray}{gray}{0.9}
%-----------------

%\usetheme{CambridgeUS}
% \usecolortheme{default}

\usetheme{default}
\usecolortheme{default}

%\usetheme{default}
%\usecolortheme{beaver}
%\usetheme{CambridgeUS}
%\usecolortheme{seagull}

%\useinnertheme{rectangles}


\setbeamertemplate{caption}{\insertcaption}	%그림의 캡션 자동 만들지 않도록.

\addtobeamertemplate{navigation symbols}{}{%
    \usebeamerfont{footline}%
    \usebeamercolor[fg]{footline}%
    \hspace{1em}%
    \insertframenumber/\inserttotalframenumber
}

\title[Types and Programming Languages]{Types and Programming Languages}
\author[K. Choi]{Kwanghoon Choi}
\institute[Chonnam National University]{
Software Languages and Systems Laboratory \\
	Chonnam National University}
\date{Week 0}

%%%%% Macros %%%%%

\newcommand{\rpc}{$\lambda_{rpc}$}
\newcommand{\polyrpc}{$\lambda_{rpc}^{\forall}$}
\newcommand{\stateencrpc}{$\lambda_{rpc}^{enc}$}
\newcommand{\statefulrpc}{$\lambda_{rpc}^{state}$}

\newcommand{\cs}{$\lambda_{cs}$}
\newcommand{\stateenccs}{$\lambda_{cs}^{enc}$}
\newcommand{\statefulcs}{$\lambda_{cs}^{state}$}

\newcommand{\client}{\textbf{c}}
\newcommand{\server}{\textbf{s}}
\newcommand{\clientserver}{\textbf{cs}}

\newcommand{\Loc}{Loc}

\newcommand{\evalRPC}[3]{#1\Downarrow_{#2}#3}
\newcommand{\evalRPCC}[2]{#1\Downarrow_{\client}#2}
\newcommand{\evalRPCS}[2]{#1\Downarrow_{\server}#2}
\newcommand{\lamL}[3]{\lambda^{#1}#2.#3}
\newcommand{\appL}[3]{#1{\ }^{#2}#3} 
\newcommand{\subst}[2]{\{#1/#2\}}
\newcommand{\llet}[3]{\textsf{let} \ #1 = #2 \ \textsf{in} \ #3}

\newcommand{\textsfReq}{\textsf{req}}
\newcommand{\req}[2]{\textsfReq(#1,#2)}
\newcommand{\reqwith}[3]{\textsfReq_{#1}(#2,#3)}

\newcommand{\textsfCall}{\textsf{call}}
\newcommand{\call}[2]{\textsfCall(#1,#2)}
\newcommand{\callwith}[3]{\textsfCall_{#1}(#2,#3)}

\newcommand{\textsfRet}{\textsf{ret}}
\newcommand{\ret}[1]{\textsfRet(#1)}
\newcommand{\retwith}[2]{\textsfRet_{#1}(#2)}

\newcommand{\funL}[1]{\xrightarrow{#1}}    
\newcommand{\funLC}[2]{\xrightarrow[#2]{#1}}  
\newcommand{\tyenv}{\Gamma}     
\newcommand{\tyenvExt}[2]{\Gamma\{#1:#2\}}
\newcommand{\tyenvExtWith}[1]{\Gamma,#1}
% \newcommand{\typing}[4]{#1\rhd_{#2} #3 : #4}       
\newcommand{\typing}[4]{#1\vdash_{#2} #3 : #4}       
\newcommand{\typingBlack}[4]{#1\blacktriangleright_{#2} #3 : #4}  

\newcommand{\loceta}[2]{{#1}\rightsquigarrow{#2}}                                         

\newcommand{\enc}{\textsf{enc}}
\newcommand{\evalStateEncRPCC}[2]{$#1\Downarrow_{\client}^{\enc}#2$}
\newcommand{\evalStateEncRPCS}[3]{$#1;#2\Downarrow_{\server}^{\enc}#3$}

\newcommand{\sta}{\textsf{state}}
\newcommand{\evalStatefulRPCC}[3]{\evalStatefulRPC{#1}{#2}{\client}{#3}}
\newcommand{\evalStatefulRPCS}[3]{\evalStatefulRPC{#1}{#2}{\server}{#3}}
\newcommand{\evalStatefulRPC}[4]{${#1};{#2}\Downarrow_{#3}^{\sta}{#4}$}

\newcommand{\deep}{\textsf{deep}}
\newcommand{\evalDeeplyStatefulRPCC}[3]{\evalDeeplyStatefulRPC{#1}{#2}{\client}{#3}}
\newcommand{\evalDeeplyStatefulRPCS}[3]{\evalDeeplyStatefulRPC{#1}{#2}{\server}{#3}}
\newcommand{\evalDeeplyStatefulRPC}[4]{${#1};{#2}\Downarrow_{#3}^{\deep}{#4}$}

\newcommand{\IdK}{\textsf{Id}}
\newcommand{\FunK}[3]{\textsf{Fun} \ #2 \ #3}
\newcommand{\AppK}[3]{\textsf{App} \ #2 \ #3}
%\newcommand{\FunK}[3]{\textsf{Fun}^{#1} \ #2 \ #3}
%\newcommand{\AppK}[3]{\textsf{App}^{#1} \ #2 \ #3}
                                                
\newcommand{\reify}[1]{\ulcorner #1 \urcorner}

\newcommand{\RightarrowEnc}{\Rightarrow^{enc}}
\newcommand{\RightarrowEncStar}{\Rightarrow^{enc*}}
\newcommand{\RightarrowEncPlus}{\Rightarrow^{enc+}}

\newcommand{\runStateEncRPC}[2]{$#1 \RightarrowEnc #2$}
\newcommand{\runStateEncRPCStar}[2]{$#1 \RightarrowEncStar #2$}
\newcommand{\runStateEncRPCPlus}[2]{$#1 \RightarrowEncPlus #2$}

\newcommand{\runStateEncCS}[2]{$#1 \Rightarrow^{enc} #2$}
\newcommand{\runStateEncCSStar}[2]{$#1 \Rightarrow^{enc*} #2$}

\newcommand{\runStatefulRPC}[2]{$#1 \Rightarrow^{state} #2$}
\newcommand{\runStatefulRPCStar}[2]{$#1 \Rightarrow^{state*} #2$}

\newcommand{\runStatefulCS}[2]{$#1 \Rightarrow^{state} #2$}
\newcommand{\runStatefulCSStar}[2]{$#1 \Rightarrow^{state*} #2$}

\newcommand{\emp}{\epsilon}
\newcommand{\substzsxs}{\{\bar{v}/\bar{z},\overline{w}/\bar{x} \}}
\newcommand{\substxs}{\{\overline{w}/\bar{x} \}}

\newcommand{\substZsXs}{\{\overline{V}/\bar{z},\overline{W}/\bar{x} \}}
\newcommand{\substXs}{\{\overline{W}/\bar{x} \}}

\newcommand{\LetK}[2]{\textsf{ctx}\ #1 \ #2}
%\newcommand{\LetK}[2]{(#1,#2)}
\newcommand{\opt}[1]{#1_{opt}}

\newcommand{\overlineK}{\overline{K}}
\newcommand{\overlinePi}{\overline{\Pi}}
\newcommand{\overlineDelta}{\overline{\Delta}}

%\newcommand{\comp}[1]{\rightsquigarrow_{#1}}
%\newcommand{\comps}{\rightsquigarrow_{\server}}
%\newcommand{\compc}{\rightsquigarrow_{\client}}
\newcommand{\ccomp}[1]{C[\![#1]\!]}
\newcommand{\scomp}[1]{S[\![#1]\!]}
\newcommand{\vcomp}[1]{V[\![#1]\!]}
\newcommand{\cconv}[1]{CC[\![#1]\!]}
%\newcommand{\cconvprg}[1]{CC_{prg}[\![#1]\!]}

\newcommand{\FUNS}{\Phi}
\newcommand{\fv}[1]{\textsf{fv}(#1)}
\newcommand{\dom}[1]{\textsf{dom}(#1)}
\newcommand{\clo}[2]{clo({#1},{#2})}

\newcommand{\sessionNothing}{\makebox[0.3cm][c]{\scriptsize $nothing$}}
\newcommand{\sessionSomething}{\makebox[0.3cm][c]{\scriptsize $session$}}
\newcommand{\sessionOption}{\makebox[0.3cm][c]{\scriptsize $optSession$}}

\newcommand{\mono}[1]{[\![#1]\!]}

\newcommand{\eqdef}{\overset{\mathrm{def}}{=\joinrel=}}

%%
%\newtheorem{lemma}{Lemma}[section]
%\newtheorem{theorem}{Theorem}[section]
%\newtheorem{fact}{Fact}[section]
%\newtheorem{definition}{Definition}[section]

%%%%%%%%%%%%%%%%%%

\begin{document}

%\section{Relative Pronoun}
\begin{frame}
%\begin{center}
%{\footnotesize
%Types and Programming Languages
%}
%\end{center}
	\titlepage
	
%	\begin{center}
%	{
%  (Joint work with James Cheney, Simon Fowler, and Sam Lindley)
%	}
%	\end{center}
\end{frame}

%\begin{frame}{Table of Contents}
%\begin{itemize}
%\item
%\item
%\item
%\item
%\end{itemize}
%\end{frame}

%%%
\begin{frame}[t]{This class} \vspace{10pt}

The study of type systems and and of programming languages from a type-theoretic perspective

\vspace{10pt}

Potential audiences 
\begin{itemize}
\item mature undergraduates/graduates from all areas of computer science who want an introduction to key concepts in the theory of programming languages
% \item \st{graduate students specializing in programming languages and type theory}
\end{itemize}

\vspace{10pt}

An extensive introductory material and a wealth of examples, exercises, and case studies. 

\end{frame}

%%%
\begin{frame}[t]{Materials} \vspace{10pt}

Textbook
\begin{itemize}
\item Types and Programming Languages by Benjamin C. Pierce
\item http://www.cis.upenn.edu/$\sim$bcpierce/tapl
\end{itemize}

\vspace{10pt}

Typecheckers present in this course are available in concrete implementations written in O'Caml 

\end{frame}

%%%
\begin{frame}[t]{Goals: Theory} \vspace{10pt}

Core topics
\begin{itemize}
\item Basic operational semantics for programming languages
\item The untyped lambda calculus
\item Simple type systems
\item Associated proof techniques to show its type safety
\item Simple extensions: the unit type, let bindings, pairs, tuples, records, sums, variants, general recursion, lists
\item More extensions: references, exceptions.
\item Extended type systems for the extended lambda calculi 
\item To prove the extended type safety using the same proof techniques
\end{itemize}

\end{frame}

%%%
\begin{frame}[t]{Required Background} \vspace{10pt}

We assume no preparation in the theory of programming languages.

\vspace{10pt}

But readers should start with a degree of mathematical maturity - in particular, rigorous undergraduate coursework in discrete mathematics, algorithms, and elementary logic

\vspace{10pt}

It would be good for readers to be familiar with at least one higher-order functional programming language (Scheme, ML, Haskell, etc.), and with basic concepts of programming languages and compilers (abstract syntax, BNF grammars, evaluation, abstract machine, etc.)

\end{frame}

%%%
\begin{frame}[t]{Goal: Practice} \vspace{10pt}

As well as the theoretical topics, students will learn a functional programming language called O'Caml. 

\vspace{10pt}

O'Caml is a functional programming language with object-oriented styles.
\begin{itemize}
\item https://ocaml.org
\end{itemize}

\vspace{10pt}

Tutorial: Basic language concepts and how to install
\begin{itemize}
\item https://dev.realworldocaml.org/
\end{itemize}

\vspace{10pt}

No required preliminary background is needed.

\end{frame}


\end{document}

\begin{frame}[t,fragile]{Introduction} \vspace{10pt}

The  RPC calculus, the simple semantics foundation for the tierless programming languages (Cooper\&Wadler, 2009)
\begin{itemize}
\item  Uses the syntax of $\lambda$-application for remote procedure calls
\end{itemize}

\begin{lstlisting}[escapechar=\%,language=lisp]
fun main() client { %\underline{authenticate}% () }

fun authenticate() server { 
 var creds = %\underline{getCredentials}%( "Enter name:passwd > " )
 if ( creds=="ezra:opensesame" ) { "The secret document" }
 else { "Access denied" }
}

fun getCredentials(prompt) client {
 (print(prompt);  read) 
}
\end{lstlisting}

cf. The other tierless PLs (ML5, Hop, Ur/Web, Eliom, etc.)
support asymmetric communication, or they are for peer-to-peer model.

\end{frame}

%%%
%\begin{frame}[t]{In this research} \vspace{10pt}
% 
%A theory of RPC calculi for client-server model
%\begin{itemize}
%\item Static identification of remote procedure calls % \\ (cf. Dynamic identification of RPCs)
%\item Type-based slicing into the client and server parts % \\ $~$ \ \ stateless strategy and stateful strategy
%\end{itemize}
%\vspace{10pt}
%
%\begin{center}
%\includegraphics[width=0.9\textwidth]{overview_typedrpconly}
%\end{center}
%
%\end{frame}

%%%
\begin{frame}[t]{The  RPC calculus (Cooper\&Wadler, 2009)} \vspace{10pt}

A call-by-value $\lambda$-calculus with location annotated $\lambda$-abstractions \\
\begin{itemize}
\item $\lambda^a x.N$ :  $\lambda$-abstraction that must run at location $a$
\item $\evalRPC{M}{a}{V}$ : A term $M$ at location $a$ evaluates to $V$
\end{itemize}
\vspace{10pt}

\begin{tabular}{l l l c l c l }
   & Location & $a,b$ 	& $::=$ & $\client$	& $|$		& $ \server$		 \\ \\
   & Term     & $L,M,N$	& $::=$ & $V$		& $ |$	& $L \ M$			 \\
   & Value    & $V,W$	& $::=$ & $x$		& $ |$	& $ \lambda^a x.N$
\end{tabular} 
\vspace{10pt}

\begin{tabular}{l l l c l c l }
  & \multicolumn{6}{l}{Evaluation} \\
  & \multicolumn{6}{c}{
  	\mbox{
       \begin{prooftree}
       		\infer0[{\footnotesize (Abs)}]{ \evalRPC{\lamL{b}{x}{M}}{a}{\lamL{b}{x}{M}} }
	\end{prooftree}
  	}
    } \\[0.4cm]
  & \multicolumn{6}{c}{
  	\mbox{
	\begin{prooftree}
  		\hypo{ \evalRPC{L}{a}{\lamL{b}{x}{N}} }
  		\hypo{ \evalRPC{M}{a}{W} }
  		\hypo{ \evalRPC{N\subst{W}{x}}{b}{V}  }
  		\infer3[{\footnotesize (App)}]{ \evalRPC{L \ M}{a}{V}  }
	\end{prooftree} 
  	}
    }     
\end{tabular}

\end{frame}

\begin{frame}[t]{The typed RPC calculus (Choi\&Chang 2019)} \vspace{10pt}

Every $\lambda$-abstraction  of type $A\funL{a}B$  runs at  location $a$. 

\vspace{10pt}

\begin{tabular}{l l l c l c l }
   & Type     & $A,B$ & $::=$ & $base$ & $ | $ & $ A\funL{a}A$ 
\end{tabular} 

\vspace{20pt}

\begin{itemize}
\item 
$
(\lamL{ \server }{f}{ \ (\lamL{\server}{x}{x}) \  ({\color{red} f} \ M)  }) \ \ (\lamL{\client}{y}{  \ (\lamL{\server}{z}{z}) \ \ y   })
$
\\[0.3cm]
Well-typed where $f : A\funL{{\color{red} \client}}B$ % {\color{red} at the server}. \\ 
% In other word, `$f \ M$' is a remote procedure call (RPC).

\vspace{20pt}

\item
$
(\ \lamL{\client}{f}{ \ {\color{red} f } \ M } \ ) \ \ 
	(\sf{if} \ \cdots 
		\ \sf{then} \ \lamL{\client}{x}{M_1} 
		\ \sf{else} \ \lamL{\server}{y}{M_2})
$
\\[0.3cm]
Ill-typed because neither $f  : A\funL{\client}B$ nor $f  : A\funL{\server}B$ 
\end{itemize}
 

\end{frame}

%%%
\begin{frame}[t]{The typed RPC calculus (cont.)} \vspace{10pt}

%A locative type system for the RPC calculus  that can identify remote procedure calls  statically 
%\vspace{20pt}

A typing judgment, $\typing{\tyenv}{a}{M}{A}$, says:
\begin{itemize}
\item A term $M$ at location $a$ has type $A$ under a type environment $\tyenv$ 
\end{itemize}

\vspace{20pt}

Key idea: A refinement of the lambda application typing w.r.t. the combinations of  location $a$ and  location $b$ 

\begin{center}
 \mbox{
	\begin{prooftree}
		\hypo{ \typing{\tyenv}{ {\color{red} a} }{L}{A\funL{ {\color{red} b} }B} }
		\hypo{ \typing{\tyenv}{ {\color{red} a} }{M}{A}  }
		\infer2{ \typing{\tyenv}{ {\color{red} a} }{L \ M}{B} }
	\end{prooftree}  
  }
\end{center}

\vspace{10pt}

cf. 
  	\mbox{
	\begin{prooftree}
  		\hypo{ \evalRPC{L}{a}{\lamL{b}{x}{N}} }
  		\hypo{ \evalRPC{M}{a}{W} }
  		\hypo{ \evalRPC{N\subst{W}{x}}{b}{V}  }
  		\infer3[{\footnotesize (App)}]{ \evalRPC{L \ M}{a}{V}  }
	\end{prooftree} 
  	}
  	
 \vspace{10pt}
 The key idea is essential for the type-directed slicing compilations in the typed RPC calculus to distinguish RPC from L(ocal)PC.

\end{frame}

%%%
%\begin{frame}[t]{Background} \vspace{10pt}
%
%Only the RPC calculus supports symmetric communication programming  for client-server model.
%\begin{itemize}
%\item The other tierless PLs (ML5, Hop, Eliom, etc.)
%support asymmetric communication, or they are for peer-to-peer model.
%\end{itemize}
%\vspace{10pt}
%
%The RPC calculus supports only the stateless server style by 
%the complex slicing compilation rules.
%\begin{itemize}
%\item due to the lack of location info. in application expressions
%\end{itemize}
%\vspace{10pt}
%
%The typed RPC calculus (Choi\&Chang, 2019) supports both the stateless and the stateful server style 
%\begin{itemize}
%\item by the simple slicing compilation rules exploiting type-based location information
%\end{itemize}
%
%\end{frame}

%%%
\begin{frame}[t]{Problem} \vspace{10pt}

However, the typed RPC calculus does not support polymorphic locations.
\vspace{10pt}

For example, we may want to write $map$ as:
	 \[
	 	(\Lambda l. \lamL{l}{f}{ \lamL{l}{xs}{\ \cdots}}) \ : \ \forall l. (A \funL{l} B) \funL{l} ([A] \funL{l} [B])
	\] 

to instantiate it as a client function by $map [\client]$, which becomes 
	\[
		\lamL{\client}{f}{ \lamL{\client}{xs}{\ \cdots}} \ : \  (A \funL{\client} B) \funL{\client} ([A] \funL{\client} [B])
	\]

\vspace{10pt}

$\Rightarrow$ A technical challenge: The key idea of  the typed RPC calculus becomes useless due to the presence of location variables $l$. 

\end{frame}

%%%
\begin{frame}[t]{A solution} 
 
A parametric polymorphism over locations for programmers to use
\begin{itemize}
\item location abstraction ($\Lambda l. V$) and
\item location application ($M[l]$)
\end{itemize}
\vspace{10pt}

A translation of polymorphic locations into monomorphic ones for compiler writers to utilize the existing slicing compilations
\vspace{10pt}

How?  by interpreting
\begin{itemize}
\item location abstraction as a pair of client/server instances, and
\item location application as a projection from the pair
\end{itemize}
\vspace{10pt}

For example,
\vspace{-15pt}
\begin{eqnarray*}
id= \Lambda l.\lamL{l}{x}{x} & \Rightarrow & \mono{id}=(\lamL{\client}{x}{x}, \lamL{\server}{x}{x}) \\
id[\client] & \Rightarrow & \pi_1(\mono{id})\\
id[\server] & \Rightarrow & \pi_2(\mono{id}) 
\end{eqnarray*}

\end{frame}

%%%
\begin{frame}[t]{In this research} \vspace{10pt}
 
An extension of the typed RPC calculus with the notion of polymorphic location 
\vspace{10pt}

\begin{itemize}
\item A polymorphic RPC calculus that supports a parametric polymorphism over locations for programmers
\vspace{5pt}

\item A monomorphization translation of the polymorphic RPC calculus into the typed RPC calculus for compiler writers
\vspace{5pt}

\item Type soundness of the polymorphic location type system and the type/semantics correctness of the translation
\vspace{5pt}
\end{itemize}

\end{frame}


%%%%
%\begin{frame}[t]{Related Work} \vspace{10pt}
%
%Symmetric communication for peer-to-peer model
%\begin{itemize}
%\item Lambda5 (Murphy VII et al, 2004), ML5 (Murphy, 2008)
%: Modal logic based calculus 
%\vspace{5pt}
%
%\item A multi-tier calculus (Neubauer \& Thiemann, 2005) 
%: Automatic compilation of RPC with session types
%\vspace{5pt}
%\end{itemize}
%
%Symmetric communication for client-server model 
%\begin{itemize}
%\item Links (Cooper et al, 2007), RPC (Cooper\&Wadler, 2009)
%\vspace{5pt}
%\end{itemize}
%
%Asymmetric communication for client-server model
%\begin{itemize}
%\item Hop (Serrano \& Queinnec, 2010)
%\vspace{5pt}
%
%\item Ur/Web (Chlipala, 2015) : A dependently typed calculus
%\vspace{5pt}
%
%\item Eliom (Radanne, 2017)  : Module with location
%\vspace{5pt}
%\end{itemize}
%
%\end{frame}

%%%
\begin{frame}[t]{Overview} \vspace{10pt}

A polymorphically locative type system with a monomoprhization translation for an extension of the typed RPC calculus
\begin{center}
\includegraphics[width=\textwidth]{polyrpc_figure}
\end{center}

\end{frame}

%%%
\begin{frame}[t]{Part I: A polymorphic  RPC calculus} \vspace{10pt}

\hspace*{-0.5cm}
\begin{tabular}{l l l c l c l c l c l }
   & Location & $a,b$ 	& $::=$ & $\client$	& $|$		& $ \server$		 \\ 
	&	                  & $\Loc$   	& $::=$  	& $a$              	& $|$          & $l$   \\
   & Term     & $L,M,N$	& $::=$ & $V$		& $ |$	& $L \ M$	&	$ |$	& $M[\Loc]$ & $|$	& $M[A]$		 \\
   & Value    & $V,W$	& $::=$ & $x$		& $ |$	& $ \lamL{\Loc}{x}{N}$ &		$|$		& $\Lambda l.V$ 	& $|$	& $\Lambda\alpha.V$
\end{tabular} 
\vspace{10pt}

\hspace*{-0.5cm}
\begin{tabular}{l l l c l c l }
  & \multicolumn{6}{l}{Evaluation} \\
  & & \multicolumn{5}{c}{
  	\mbox{
       \begin{prooftree}
       		\infer0[{\footnotesize (Abs)}]{ \evalRPC{\lamL{b}{x}{M}}{a}{\lamL{b}{x}{M}} }
	\end{prooftree}
  	}
    } \\[0.5cm]
  & & \multicolumn{5}{c}{
  	\mbox{
	\begin{prooftree}
  		\hypo{ \evalRPC{L}{a}{\lamL{b}{x}{N}} }
  		\hypo{ \evalRPC{M}{a}{W} }
  		\hypo{ \evalRPC{N\subst{W}{x}}{b}{V}  }
  		\infer3[{\footnotesize (App)}]{ \evalRPC{L \ M}{a}{V}  }
	\end{prooftree} 
  	}
    }  \\[0.5cm]
    & & \multicolumn{5}{c}{
    	\mbox{
    		\begin{prooftree}
       		\infer0[{\footnotesize(Labs)}]{ \evalRPC{\Lambda l.V}{a}{\Lambda l.V} }
	\end{prooftree}
	\ \ \ 
	\begin{prooftree}
  		\hypo{ \evalRPC{M}{a}{\Lambda l.{V}} }
    		\infer1[{\footnotesize(Lapp)}]{ \evalRPC{M[b]}{a}{V\subst{b}{l}}  }
	\end{prooftree} 

    	}
    }   
\end{tabular}
\vspace{10pt}

cf. (Tabs) and (Tapp)

\end{frame}


%%%
%%%
\begin{frame}[t]{A type system for the polymorphic  RPC calculus} \vspace{10pt}

\hspace*{-1cm}
\begin{tabular}{l l l c l c l c l c l c l }
& Type & $A,B$ & $::=$ & $base$ & $ | $ & $ A\funL{\Loc}A$  & $ | $ & $ \forall l.A$ & $|$ & $\alpha$ & $|$ & $\forall\alpha.A$ 
\end{tabular} 
\vspace{10pt}

\hspace*{-1cm}
\begin{tabular}{l l l c l c l }
  & \multicolumn{6}{l}{Typing rules} \\
  & & \multicolumn{5}{c}{
  \mbox{
  	\begin{prooftree}
		\hypo{  \tyenv(x)=A }
		\infer[left label={\footnotesize (T-Var)}]1{ \typing{\tyenv}{\Loc}{x}{A} }
	\end{prooftree}
	\ \ \ \ \ 
	\begin{prooftree}
		\hypo{ \typing{\tyenvExt{x}{A}}{\Loc}{M}{B} }
		\infer[left label={\footnotesize(T-Abs)}]1{ \typing{\tyenv}{\Loc'}{\lamL{\Loc}{x}{M}}{A\funL{\Loc}B} } 
	\end{prooftree}
    } 
    } \\[0.7cm]
  & & \multicolumn{5}{c}{
 \mbox{
	\begin{prooftree}
		\hypo{  \typing{\tyenv}{\Loc}{L}{A\funL{\Loc'}B } }
		\hypo{  \typing{\tyenv}{\Loc}{M}{A} }
		\infer[left label={\footnotesize(T-App)}]2{ \typing{\tyenv}{\Loc}{L \ M}{B}   }
	\end{prooftree}
	}
    }  \\[0.7cm]
    & & \multicolumn{5}{c}{
  \mbox{
	\begin{prooftree}
		\hypo{ \typing{\tyenvExtWith{l}}{\Loc}{V}{A} }
		\infer[left label={\footnotesize(T-Labs)}]1{ \typing{\tyenv}{\Loc}{\Lambda l.V}{\forall l.A }} 
	\end{prooftree}
	\ \ \ \ \ 
	\begin{prooftree}
		\hypo{ \typing{\tyenv}{\Loc}{M}{\forall l.A } }
		\infer[left label={\footnotesize(T-Lapp)}]1{ \typing{\tyenv}{\Loc}{M[\Loc']}{A\subst{\Loc'}{l}}} 
	\end{prooftree}
    } 


    }   
\end{tabular}
\vspace{10pt}

cf. (T-Tabs) and (T-Tapp)

\end{frame}

%%%
\begin{frame}[t]{Properties of the polymorphic type system} \vspace{10pt}

Type soundness for the polymorphic RPC calculus
\begin{itemize} 
\item  For a closed term $M$, if \ $\typing{\emptyset}{a}{M}{A}$ and $\evalRPC{M}{a}{V}$, then $\typing{\emptyset}{a}{V}{A}$.
\end{itemize}

\vspace{10pt}

No lambda abstraction with any location variable will be a value during evaluation on closed terms.

\end{frame}

%%%%
\begin{frame}[t]{Part II: A monomorphization translation of {\polyrpc} into {\rpc}} 

A basic idea: location abstraction as a pair of client/server instances, and location application as its projection
\vspace{7pt}

\begin{tabular}{l l l l l l l l l }
\multicolumn{9}{l}{Translation: types} \\
\ \ 
	$\mono{base}=base$  & 
\\[0.2cm]
\ \ 
	$\mono{\alpha}=\alpha$ &
	$\mono{\forall\alpha.A}=\forall\alpha.\mono{A}$
\\[0.2cm]
\ \ 
	{\color{blue} $\mono{A \funL{a}B}=\mono{A}\funL{a}\mono{B}$ } & 
	{\color{red} $\mono{\forall l.A} = \mono{ A\subst{\client}{l} } \times \mono{A\subst{\server}{l} }$ }
\\[0.5cm]
\multicolumn{9}{l}{Translation: terms} \\
\ \ 
	$\mono{x}=x$ &
	{\color{blue}  $\mono{\lamL{a}{x}{M}} = \lamL{a}{x}{\mono{M}}$ } &
\\[0.2cm]
\ \ 
	$\mono{L \ M} = \mono{L} \ \mono{M}$ &
\\[0.2cm]
\ \ 
	$\mono{\Lambda\alpha.V}=\Lambda\alpha.\mono{V}$ &
	$\mono{M[B]}=\mono{M}[\mono{B}]$
\\[0.2cm]
\multicolumn{9}{l}{\ \ {\color{red} $\mono{\Lambda l. V} = (\mono{V\subst{\client}{l}}, \mono{V\subst{\server}{l}})$ }}
\\[0.2cm]
\ \ 
	{\color{red} $\mono{M[\client]}  = \pi_1(\mono{M})$ } &
	{\color{red} $\mono{M[\server]} = \pi_2(\mono{M})$ }
\end{tabular}

\end{frame}

%%%
\begin{frame}[t]{Example (1)} 

The identity function:
	\begin{eqnarray*}
		\mono{\forall l. \forall\alpha.\alpha\funL{l}\alpha} 
		&=& (\mono{(\forall\alpha.\alpha\funL{l}\alpha)\subst{\client}{l}}, \ \mono{(\forall\alpha.\alpha\funL{l}\alpha)\subst{\server}{l}})  \\
		&=& (\mono{(\forall\alpha.\alpha\funL{\client}\alpha)}, \ \mono{(\forall\alpha.\alpha\funL{\server}\alpha)}) \\
		&=& (\forall\alpha.\mono{(\alpha\funL{\client}\alpha)}, \ \forall\alpha.\mono{(\alpha\funL{\server}\alpha)}) \\
		&=& (\forall\alpha.\mono{\alpha}\funL{\client}\mono{\alpha}, \ \forall\alpha.\mono{\alpha}\funL{\server}\mono{\alpha}) \\
		&=& (\forall\alpha.\alpha\funL{\client}\alpha, \ \forall\alpha.\alpha\funL{\server}\alpha) 
\\
\\
		\mono{\Lambda l. \Lambda\alpha.\lamL{l}{x}{x}} 
		&=& (\mono{(\Lambda\alpha.\lamL{l}{x}{x})\subst{\client}{l}}, \ \mono{(\Lambda\alpha.\lamL{l}{x}{x})\subst{\server}{l}})  \\
		&=& (\mono{\Lambda\alpha.\lamL{\client}{x}{x}}, \ \mono{\Lambda\alpha.\lamL{\server}{x}{x}})  \\
		&=& (\Lambda\alpha.\mono{\lamL{\client}{x}{x}}, \ \Lambda\alpha.\mono{\lamL{\server}{x}{x}})  \\
		&=& (\Lambda\alpha.\lamL{\client}{x}{\mono{x}}, \ \Lambda\alpha.\lamL{\server}{x}{\mono{x}}) \\
		&=& (\Lambda\alpha.\lamL{\client}{x}{x}, \ \Lambda\alpha.\lamL{\server}{x}{x}). 
	\end{eqnarray*}

\end{frame}

%%%
\begin{frame}[t]{Example (2)} 

%Assume $M_X$ is a body of a map function parameterized by location arguments $X$ as 

A map function of type $\forall l. (A \funL{l} B) \funL{l} ([A] \funL{l} [B])$: 
\[
	\texttt{ letrec } \ map = \Lambda{l}.\lamL{l}{f}{ \lamL{l}{xs}{ M_{l} } } \ \texttt{ in } \ \cdots
\]
where 
$
M_X \eqdef case \ xs \ of \ \{ [] \Rightarrow []; \ (y:ys) \Rightarrow f \ y ::  map [X] \ f \ ys \}
$
	\begin{eqnarray*}
	\mono{map} &=& ( \mono{ (\lamL{l}{f}{ \lamL{l}{xs}{ M_{l}} }) \subst{\client}{l} } ,  \ \ \mono{ (\lamL{l}{f}{ \lamL{l}{xs}{ M_{l}} }) \subst{\server}{l} } ) \\
	&=& ( \mono{ \lamL{\client}{f}{ \lamL{\client}{xs}{ M_{\client}} } } , \ \ \mono{\lamL{\server}{f}{ \lamL{\server}{xs}{ M_{\server}} } } ) \\
	&=& ( \lamL{\client}{f}{ \mono{\lamL{\client}{xs}{ M_{\client}} } } , \ \ \lamL{\server}{f}{ \mono{\lamL{\server}{xs}{ M_{\server}} } } ) \\
	&=& ( \lamL{\client}{f}{ \lamL{\client}{xs}{ \mono{M_{\client}} } } , \ \ \lamL{\server}{f}{ \lamL{\server}{xs}{ \mono{M_{\server}} } } ).
	\end{eqnarray*} 
where 
\begin{itemize}
\item $\mono{map[\client]}$ in $\mono{M_{\client}}$ is translated as $\pi_1(\mono{map})$, and 
\item $\mono{map[\server]}$ in $\mono{M_{\server}}$ is translated as $\pi_2(\mono{map})$.
\end{itemize}

\end{frame}

%%%
\begin{frame}[t]{Example (3)} 

A map function of type $\forall l_1.\forall l_2.\forall l_3. (A \funL{l_3} B) \funL{l_1} ([A] \funL{l_2} [B])$: 
\[
\texttt{ letrec } \ map = \Lambda{l_1}.\Lambda{l_2}.\Lambda{l_3}.map_0  \ \texttt{ in } \ \cdots
\]
where  $map_0 \ = \ \lamL{l_1}{f}{ \lamL{l_2}{xs}{ \ M_{l_1 l_2 l_3} }}$
\vspace{10pt}

$\mono{map}$ is translated as
\[
		\left(
	\mbox{
		\begin{tabular}{l }
		$\left( 
			\mbox{
			\begin{tabular}{l }
			$( \mono{map_0\{\client/l_1,\client/l_2,\client/l_3\}},  \mono{map_0\{\client/l_1,\client/l_2,\server/l_3\}} ),$ \\
			$( \mono{map_0\{\client/l_1,\server/l_2,\client/l_3\}},  \mono{map_0\{\client/l_1,\server/l_2,\server/l_3\}} )$
			\end{tabular}
			}
		\right),$ \\ 
		$\left( 
			\mbox{
			\begin{tabular}{l }
			$( \mono{map_0\{\server/l_1,\client/l_2,\client/l_3\}},  \mono{map_0\{\server/l_1,\client/l_2,\server/l_3\}} ),$ \\
			$( \mono{map_0\{\server/l_1,\server/l_2,\client/l_3\}},  \mono{map_0\{\server/l_1,\server/l_2,\server/l_3\}} )$
			\end{tabular}
			}
		\right)$
		\end{tabular}
	}
	\right)
\]
\vspace{5pt}

The monomorphization translation could generate many instances in theory, but  the first map function would be common in practice. 

\end{frame}

%%%
\begin{frame}[t]{Properties of the monomorphization translation} \vspace{10pt}

Type correctness of the translation
\begin{itemize} 
\item  For a closed term $M$, if \ $\typing{\emptyset}{a}{M}{A}$ in {\polyrpc} then $\typing{\emptyset}{a}{\mono{M}}{\mono{A}}$ in {\rpc}.
\end{itemize}

\vspace{10pt}

Semantic correctness of the translation
\begin{itemize} 
\item For a closed term $M$, if \ $\typing{\emptyset}{a}{M}{A}$ and $\evalRPC{M}{a}{V}$ in {\polyrpc}, then $\evalRPC{\mono{M}}{a}{\mono{V}}$ in {\rpc}.
\end{itemize}

\end{frame}

%%%
\begin{frame}[t]{Conclusion} 

The polymorphic RPC calculus 
\begin{itemize}
\item provides a parametric polymorphism over locations, and 
\item can utilize the existing type-directed slicing compilations.
\end{itemize}

\begin{center}
\includegraphics[width=1\textwidth]{polyrpc_figure}
\end{center}

%\begin{center}
%\includegraphics[width=\textwidth]{loch_katrine}
%\end{center}

\end{frame}

\end{document}

